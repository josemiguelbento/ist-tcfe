\section{Theoretical Analysis}
\label{sec:analysis}

In this section, the circuit shown in \textbf{Figure~\ref{fig:diagram_t3}} is analysed
theoretically.
\begin{figure}[H] \centering
\includegraphics[width=0.6\linewidth]{diagram_t3.pdf}
\caption{Diagram of the circuit considered for the computations and simulations}
\label{fig:diagram_t3}
\end{figure}

In the Envolope Detector the diode model used has an ideal diode and an voltage source while in the Voltage regulator de Diode model has a ideal diode, a voltage source and a resistor. 

The output of the transformer will be the input of our Envelope detector.The Envelope detector circuit is made up of a rectifier and a capacitor.The rectifier is a full-wave bridge rectifier circuit and is composed of 4 diodes and a resistor. The full-wave rectifier was used as oposed of a half-wave rectifier because the full-wave rectifier allows us to reduce the ripple without incrising the time constant which would avoid a big raise in costes. The full-wave rectifier reduces the ripple because the voltage that comes out of the transformer will leave the rectifier oscillating at twice the frequency and with less amplitude, which helps us in our ripple problem.

The output of the Envelope detector    
%point 1


%In \textbf{Table~\ref{tab:theoretical}} the values for the branch currents and the node voltages obtained from the Octave script for both methods are presented. Here, the node voltages in the mesh method were computed from the respective currents, which were determined as described in the previous subsection.





%\begin{figure}[H] \centering
%\includegraphics[width=0.9\linewidth]{natural_tab.pdf}
%\caption{Natural response of $V_6$ as a function os time in the interval from [0,20] ms}
%\label{fig:natural}
%\end{figure} 



%\begin{figure}[H] \centering
%\includegraphics[width=0.9\linewidth]{freq_resp_tab.pdf}
%\caption{Graph for amplitude frequency response, in dB, of $V_c$, $V_6$ and $V_s$ for frequencies ranging from 0.1Hz to 1MHz (logarithmic scale).}
%\label{fig:freq_resp}
%\end{figure}




\pagebreak



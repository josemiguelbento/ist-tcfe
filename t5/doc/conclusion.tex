\section{Conclusion}
\label{sec:conclusion}

In this laboratory assignment, the objective of designing a Bandpass filter with an OP-AMP has been achieved.\par
The circuit created was analysed theoretically using the Octave maths tool, and by circuit simulation, using the
Ngspice tool. The theoretical results obtained differed quite a lot from those obtained in the simulation analysis. This difference obtained is due to the fact that the models used for the OP-AMP in the theoretical analysis are very different from those used by Ngspice. The OP-AMP model that was used in Ngspice is way more complex than the one implemented theoretically - one needs to look no further than the dozens of parameters considered for this model (including various capacitors), compared to the handful used in the theoretical analysis (with no complex impedances), to explain this discrepance. 

However, this time we cannot say the match between models was satisfactory. As we have already discussed in section 2, the theoretical and simulation results differ a lot. This is due mainly (as would be expected by now) to the fact that the OPAMP model used includes capacitors, which introduces, by means of additional complex impedances, two extra poles in the transfer function. Not only does this subvert the format for the phase bode plot obtained theoretically, it also affects the upper cutoff frequency, and, as a result, the central passband frequency: while, in theory, the parameters used will not give us the desired frequency (in fact it would be around 100Hz and not around 1kHz), in practice, they do! \par
At the same time, the output impedance values differ by several orders of magnitude, and the gain obtained via both methods is significantly off (around 107 dB, by simulation, vs a measly 67 dB, in theory). In fact, the only similar values, besides the cost, obviously, are the lower cutoff frequencies!\par
Due to all theses discrepancies, the merit figures are very different: the simulation merit is a whopping 0.00000684 gold medals and the theoretical merit is only 0.00000008. Evidently, this seems like a very low merit figure, given that in previous works we used to be aiming torwards the dozens or even hundreds; but we believe this is also a result of different characteristics of the circuits used (in particular, given the high, fixed and unavoidable cost of the OPAMP and the limited components, which make it hard to achieve low enough central frquency and gain deviations); and as such, we hope that this merit value will be put into perspective. 
By means of a conclusion, and given the almost "trial and error" approach which we had to take on as we tried to improve the Ngpspice, simulated merit, which was very hard to make systematic in any way, and also looking back on the previous lab assignments, we leave a parting though: \textbf{for complex components such as these, a bad model or an incomplete model is nearly worse than no model at all!}\par



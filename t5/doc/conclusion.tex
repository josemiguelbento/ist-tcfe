\section{Conclusion}
\label{sec:conclusion}

In this laboratory assignment, the objective of designing a Bandpass filter with an OP-AMP has been achieved.\par
The circuit created was analysed theoretically using the Octave maths tool, and by circuit simulation, using the
Ngspice tool. The models used for the OP-AMP in the theoretical analysis are simpler than those used by Ngspice. One needs to look no further than the dozens of parameters considered for this model (including various capacitors), compared to the handful used in the theoretical analysis (with no complex impedances). Nonetheless, the results are remakably similar between both analysis. The only noteworthy differences are the ones in the lower and upper cutoff frequencies, everywhere else only finding discrepancies lower thar 5\%.

We can therefore say the match between models was satisfactory. As we have already discussed in section 2, the theoretical and simulation results differ considerably when it comes to cutoff frequencies. This is due mainly (as would be expected by now) to the fact that the OPAMP model used includes capacitors, which introduces, by means of additional complex impedances, two extra poles in the transfer function. Not only does this subvert the format for the phase bode plot obtained theoretically, it also affects the cutoff frequencies. However, there is only a small difference in the central passband frequency between theory and simulation.

Due to the discrepancies aforementioned, the merit figures are very different: the simulation merit is a whopping 0.00120399 gold medals and the theoretical merit is only 0.00000439. Evidently, this seems like a very low merit figure, given that in previous works we used to be aiming torwards the dozens or even hundreds; but we believe this is also a result of different characteristics of the circuits used (in particular, given the high, fixed and inexorable cost of the OPAMP and the limited components, which make it hard to achieve low enough central frequency and gain deviations); and as such, we hope that this merit value will be put into perspective. 
By means of a conclusion, we had to take a "trial and error" approach as we tried to improve the Ngpspice's merit, a method that was very hard to make systematic in any way. Looking back on the previous lab assignments, we leave a parting though: \textbf{for complex components such as these, an incomplete model is not to be trusted blindly}\par



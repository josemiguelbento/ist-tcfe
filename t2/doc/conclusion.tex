\section{Conclusion}
\label{sec:conclusion}

In this laboratory assignment the objective of analysing a circuit containing multiple resistances, a capacitor and a sinusoidal voltage source $v_s$ that  varies in time has been achieved.\par
Static, time and frequency analyses have been performed both theoretically, using the Octave maths tool, and by circuit simulation, using the
Ngspice tool. The theoretical results obtained match the simulation results quite precisely - one must call attention to the fact that certain values obtained theoretically were precisely zero, whilst im some cases (the boundary conditions simulated for t=0) the corresponding simulated results were of the order of 1e-15 - we considered these results to be effectively zero, given they were multiple orders of magnitude inferior to the remaining nominal values and aproximately of the magnitude of the floating point precision for the numerical representation types used.\par
The reason for this overall satisfactory match is the fact that this is a
relatively straightforward circuit, containing only one capacitor, besides linear components - as such, any differences will be related to the model used in ngspice for the capacitor, and in particular for the transient boundary condition analysis. For more complex components and circuits, the
theoretical and simulation models could differ more (given the greater complexity of the models implemented by the simulator Ngspice, as well as the interactions between these, when compared to those studied in the theoretical classes).\par


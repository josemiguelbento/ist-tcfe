\section{Theoretical Analysis}
\label{sec:analysis}

In this section, the circuit shown in \textbf{Figure~\ref{fig:diagram_t2}} is analysed
theoretically.\par

\begin{figure}[h] \centering
\includegraphics[width=0.6\linewidth]{diagram_t2.pdf}
\caption{Diagram of the circuit considered for the computations and simulations}
\label{fig:diagram_t2}
\end{figure}


\subsection{Analysis for $t<0$} 
The nodal method is aplied to the circuit in order to determine the voltage in all nodes and the current on all branches . The nodal method aplies KVL and for $t<0$ no current passes through the capacitor, and therefore this component behaves like an open circuit.
%In \textbf{Table~\ref{tab:theoretical}} the values for the branch currents and the node voltages obtained from the Octave script for both methods are presented. Here, the node voltages in the mesh method were computed from the respective currents, which were determined as described in the previous subsection.
 \pagebreak 
\begin{table}[h]
  \centering
  \begin{tabular}{|l|r|}
    \hline    
    {\bf Name} & {\bf Node method}\\ \hline
    V0 & 0\\ \hline
V1 &  5.054818641360000\\ \hline
V2 &  4.793704691824189\\ \hline
V3 &  4.258197784082119\\ \hline
V4 & -1.934225550007436\\ \hline
V5 &  4.831047093349573\\ \hline
V6 &  5.668298372306746\\ \hline
V7 & -1.934225550007436\\ \hline
V8 & -2.905231322705667\\ \hline
Ir1 &   -2.536487650531725e-01\\ \hline
Ir2 &   -2.658975828505401e-01\\ \hline
Ir3 &    1.224881779736725e-02\\ \hline
Ir4 &  1.205310280142027\\ \hline
Ir5 &    2.658975828505403e-01\\ \hline
Ir6 &   -9.516615150888543e-01\\ \hline
Ir7 &   -9.516615150888547e-01\\ \hline
Gb &    2.658975828505403e-01\\ \hline

  \end{tabular}
  \caption{A variable that starts with "Ir" and the variable "Gb" are of type {\em current}
    and expressed in milliampere (mA); all the other variables that start with a "V" are of the type {\it voltage} and expressed in
    Volt (V).}
  \label{tab:theoretical}
\end{table}



\subsection{Equivalent resistor as seen from the capacitor terminals}

To compute the equivalent resistance as seen by C the independent source $V_c$ needs to be switched off. We do this by replacing it with a short circuit ($V_s=0$). We also replace the capacitor with a voltage source $V_x=V_6-V_8$.We use the $V_6$ and $V_8$ from the previous section beacause the voltage drop at the ends of the capacitor needs to be a continuous function (there can not be an energy spike in the capacitor). With this in mind a nodal analysis is performed in order to determine the current $I_x$ that is supplied by $V_x$.With this values we can determine $R_{eq}$ ($R_{eq}=V_x/I_x$). All this procedures were required in order to determine the time constant $\tau$ ($\tau=R_{eq}*C$). The time constant in crucial to determine the natural and forced solutions for $V_6$, which will be done in the next subsections. 

%In \textbf{Table~\ref{tab:theoretical}} the values for the branch currents and the node voltages obtained from the Octave script for both methods are presented. Here, the node voltages in the mesh method were computed from the respective currents, which were determined as described in the previous subsection.
 \pagebreak 
\begin{table}[h]
  \centering
  \begin{tabular}{|l|r|}
    \hline    
    {\bf Name} & {\bf Values for aux circuit}\\ \hline
    V0 & 0\\ \hline
V1 & 0\\ \hline
V2 & 0\\ \hline
V3 & 0\\ \hline
V4 & 0\\ \hline
V5 & 0\\ \hline
V6 &  8.573529695040000\\ \hline
V7 & 0\\ \hline
V8 & 0\\ \hline
Req & -3148.773561540000\\ \hline
$\tau$ & -3.239206e-03\\ \hline
  \end{tabular}
  \caption{A variable preceded by @ is of type {\em current}
    and expressed in milliampere (mA); other variables are of type {\it voltage} and expressed in
    Volt (V).}
  \label{tab:equivalent resistor}
\end{table}



 \pagebreak 
\begin{table}[h]
  \centering
  \begin{tabular}{|l|r|}
    \hline    
    {\bf Name} & {\bf Complex amplitude voltage}\\ \hline
    V0 & 0\\ \hline
V1 &  1\\ \hline
V2 &    9.445310207903378e-01\\ \hline
V3 &    8.329148493139050e-01\\ \hline
V4 &    3.744381306846604e-01\\ \hline
V5 &    9.524211706387480e-01\\ \hline
V6 &    5.602948262043124e-01\\ \hline
V7 &    3.744381306846604e-01\\ \hline
V8 &    5.581602943867834e-01\\ \hline

  \end{tabular}
  \caption{The table for the complex amplitudes in point 4:}
  \label{tab:equivalent resistor}
\end{table}

\begin{figure}[h] \centering
\includegraphics[width=0.9\linewidth]{natural_tab.pdf}
\caption{Graph for natural response point 3}
\label{fig:natural}
\end{figure}

\begin{figure}[h] \centering
\includegraphics[width=0.9\linewidth]{vs_v6_f_tab.pdf}
\caption{Graph for forced response point 4}
\label{fig:vs_v6_f}
\end{figure}

\begin{figure}[h] \centering
\includegraphics[width=0.9\linewidth]{theo5_tab.pdf}
\caption{point 5: Graph for time interval from -5 to 20 ms}
\label{fig:theo5}
\end{figure}

\begin{figure}[h] \centering
\includegraphics[width=0.9\linewidth]{freq_resp_tab.pdf}
\caption{point 6: Graph for amplitude frequency response of Vc, V8 and Vs for frequencies ranging from 0.1Hz to 1MHz}
\label{fig:freq_resp}
\end{figure}

\begin{figure}[h] \centering
\includegraphics[width=0.9\linewidth]{angle_tab.pdf}
\caption{point 6: Graph for phase frequency response of Vc, V6 and Vs for frequencies ranging from 0.1Hz to 1MHz}
\label{fig:angle_resp}
\end{figure}






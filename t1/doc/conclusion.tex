\section{Conclusion}
\label{sec:conclusion}

In this laboratory assignment the objective of analysing a circuit containing independent, linearly dependent voltage and current sources and other linear components (resistors) has been
achieved. Static analyses have been performed both
theoretically, using the Octave maths tool, and by circuit simulation, using the
Ngspice tool. Not only did the theoretical results obtained using different methods agree perfectly, they also matched the simulation results precisely.\par
The reason for this perfect match is the fact that this is a
straightforward circuit containing only linear components - this implies, as pointed out in the theory classes, that the theoretical
and simulation models cannot differ. For more complex components and circuits, the
theoretical and simulation models could differ (given the greater complexity of the models implemented by the simulator Ngspice, when compared to those studied in the theoretical classes); however, that is not the case for this particular circuit.\par
This lab assignment has been useful to learn and give us the chance to put into practice our knowledge of the invaluable software tools and automation procedures required for this and other types of work, from Git and Makefile to Octave and the syntax of Ngspice. In more theoretical terms, we have become more at ease with the analysis of linear circuits, and, knowing what we now know, would have made differente choices regarding the analysis even of this circuit (for example, we probably would have chosen a different node to assign to the Ground potential).


\section{Simulation Analysis}
\label{sec:simulation}
\subsection{Operating Point Analysis}
\textbf{Table~\ref{tab:op}} shows the simulated operating point results for the circuit
under analysis. The current flows considered in the theoretical section were coherent with the polarity implicitly declared when defining the circuit to be simulated in the Ngspice script.\par
As mentioned previously, in Section~\ref{sec:analysis}, we had to create a "fictional" voltage source, between node 7 and resistor 6 (providing 0V to the circuit in order not to alter the behaviour of the rest of the circuit) so as to be able to define the dependecy for the current-controlled voltage source {\it $V_c$}. This has no specific reason to be, other than the particularities of the Ngspice software. The consequences of adding this additional voltage have already been described (namely the creation of a new node) and dealt with in the previous section - see, for example, equation 6.\par

\begin{table}[h]
  \centering
  \begin{tabular}{|l|r|}
    \hline    
    {\bf Name} & {\bf Value [mA or V]} \\ \hline
    @c[i] & 0.000000e+00\\ \hline
@gb[i] & -2.91567e-04\\ \hline
@r1[i] & 2.780494e-04\\ \hline
@r2[i] & 2.915672e-04\\ \hline
@r3[i] & -1.35178e-05\\ \hline
@r4[i] & -1.22689e-03\\ \hline
@r5[i] & -2.91567e-04\\ \hline
@r6[i] & 9.488377e-04\\ \hline
@r7[i] & 9.488377e-04\\ \hline
v(1) & 5.243596e+00\\ \hline
v(2) & 4.952739e+00\\ \hline
v(3) & 4.367468e+00\\ \hline
v(4) & -1.96340e+00\\ \hline
v(5) & 4.994112e+00\\ \hline
v(6) & 5.904536e+00\\ \hline
v(7) & -1.96340e+00\\ \hline
v(8) & -2.92677e+00\\ \hline

  \end{tabular}
  \caption{Operating point. A variable preceded by @ is of type {\em current}
    and expressed in miliAmpere; other variables are of type {\it voltage} and expressed in
    Volt.}
  \label{tab:op}
\end{table}

As can already be observed, the results of the simulation coincide up to the last digit with the results of the theoretical analysis.
